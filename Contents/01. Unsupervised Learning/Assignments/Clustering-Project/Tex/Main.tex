\documentclass{article}

\usepackage{graphicx}
\usepackage{fancyhdr}
\usepackage[sorting=none]{biblatex}
\usepackage[margin=1in, top=1in]{geometry}
\usepackage{listings}
\usepackage[hidelinks]{hyperref}
\usepackage{xcolor}
\usepackage{xepersian}

\addbibresource{bibliography.bib}
\settextfont[Scale=1.3]{B-NAZANIN.TTF}
\setlatintextfont[Scale=1]{DejaVuSans}
\renewcommand{\baselinestretch}{1.5}
\pagestyle{fancy}
\fancyhf{}
\rhead{یادگیری بدون نظارت}
\lhead{مهلت تحویل پروژه: ۲۴ فروردین ۱۴۰۳}
%\rfoot{مبنا های عددی}
%\lfoot{9700000}
\renewcommand{\headrulewidth}{1pt}
\renewcommand{\footrulewidth}{1pt}



\begin{document}
	\begin{centering}
		\LARGE
		پروژه اول مبانی هوش محاسباتی \\
	\end{centering}
	\section*{مقدمه}
	در این پروژه قصد داریم تا با استفاده از الگوریتم های \lr{Clustering} که در کلاس آموختید به خوشه بندی تصاویر افراد مختلف بپردازیم. دیتاستی که در اختیار شما قرار گرفته است شامل تعدادی عکس از افراد مختلف می‌باشد. هدف این است که سیستمی طراحی و پیاده سازی کنید تا بتواند تصاویر این اشخاص را به خوبی از یکدیگر جدا کند.
	
	
	\section{فاز اول}
	در این فاز از شما میخواهیم که ابتدا به استخراج ویژگی‌ از تصاویر بپردازید و برای هر تصویر بردار ویژگی آن را به دست آورید. برای این کار حداقل دو روش را امتحان کنید و در انتهای پروژه عملکرد این دو روش با یکدیگر مقایسه کنید. (می توانید از روش‌های تدریس شده در کلاس حل تمرین استفاده کنید.)
	
	
	\section{فاز دوم}
	با استفاده از روش‌ های کاهش ابعاد دیتای خود را ترسیم کنید. (برای این بخش هم می‌توانید از روش ‌های تدریس شده در کلاس حل تمرین استفاده کنید.)
	
	\section{فاز سوم}
	با توجه به نتایج فاز قبل، الگوریتم مناسب برای خوشه بندی را انتخاب کرده و پیاده سازی کنید. دلایل انتخاب این الگوریتم را نیز بیان کنید، در مرحله آخر این فاز به بررسی عملکرد خوشه‌بندی خود با تصویر سازی یا \lr{visual} کردن نتایج به دست آمده از آن بپردازید. نشان دهید الگوریتم انتخاب شده توسط شما، هم از لحاظ نوع الگوریتم هم از لحاظ مقدار هایپرپارامترها یک انتخاب خوب بوده است. (استفاده از یک روش خلاقانه با عملکرد خوب بیشترین نمره را دریافت خواهد کرد.)
	
	\section{فاز چهارم}
	در فاز چهارم و فاز آخر باید به ارزیابی مدل خود بپردازید. 
	\subsection{بخش اول}
	در این بخش انتظار داریم عکس‌ ها‌یی که در انتها لیبل یکسانی را دریافت کرده‌اند، در یک پوشه قرار دهید، سپس گزارشی از تعداد عکس ‌های هر پوشه و میزان شباهت عکس ‌ها به یکدیگر ارائه کنید.
	
	\subsection{بخش دوم}
	برای اینکه ارزیابی از نحوه عملکرد مدل خود داشته باشید، لیبل تصاویر به شما داده شده است. حال با استفاده از لیبل واقعی تصاویر و الگوریتم \lr{Rand Index} عملکرد مدل خود را مورد بررسی قرار دهید. (پیاده سازی الگوریتم \lr{Rand Index} باید توسط خودتان انجام شود.)
	
	\subsection*{توضیحات تکمیلی}
	\begin{itemize}
		\item [$\bullet$] علیرغم در اختیار داشتن لیبل ها شما اجازه استفاده از آنها را برای سایر فاز‌های پروژه ندارید و صرفا باید برای ارزیابی الگوریتم خود در فاز آخر از آن استفاده کنید. 
		
		\item [$\bullet$] انجام پروژه می‌تواند در قالب گروه های دو نفره و یا به صورت انفرادی صورت گیرد.
		
		\item [$\bullet$] علاوه بر سورس کد پروژه، فایل مستندات  نیز باید آپلود شود.
		
		\item [$\bullet$] در فایل مستندات پروژه نام هر دو عضو گروه را ذکر کنید و آپلود فایل‌ها همین که توسط یکی از اعضای گروه انجام شود کافی است.
		
		\item [$\bullet$] هر گونه شباهت نامتعارف بین کد شما و کد سایر گروه‌ها و یا کدها‌ی موجود بر روی اینترنت تقلب محسوب می‌شود و نمره‌ای برای این پروژه دریافت نخواهید کرد.
		
		\item [$\bullet$] در صورت نوشتن داکیومنت تمیز (برای مثال با \lr{\LaTeX}) نمره اضافه برای شما در نظر گرفته خواهد شد.
		
		\item [$\bullet$] استفاده از کتابخانه ها و فریم ورک‌های آماده به جز مواردی که درصورت پروژه از شما خواسته شده تا پیاده سازی کنید، بلامانع است.
		
		\item [$\bullet$] فایل شامل سورس کد پروژه و مستندات را در قالب فایل \lr{zip} و با نام شماره دانشجویی خود ذخیره و ارسال نمائید.
		
				\item [$\bullet$] در صورت داشتن هرگونه سوال میتوانید با \lr{\href{https://t.me/afsharino}{\textcolor{blue}{\lr{afsharino}}}} و یا \lr{\href{https://t.me/MohMollaei}{\textcolor{blue}{\lr{MohMollaei}}}} در ارتباط باشید.
				\newline
	\end{itemize}
	
	\begin{LTR}
		
		تمرین حل تیم - احترام با
	\end{LTR}
		

	
\end{document}
 
