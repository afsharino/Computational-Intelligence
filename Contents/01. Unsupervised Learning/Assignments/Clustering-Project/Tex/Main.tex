\documentclass{article}

\usepackage{graphicx}
\usepackage{fancyhdr}
\usepackage[sorting=none]{biblatex}
\usepackage[margin=1in, top=1in]{geometry}
\usepackage{listings}
\usepackage[hidelinks]{hyperref}
\usepackage{xcolor}
\usepackage{xepersian}

\addbibresource{bibliography.bib}
\settextfont[Scale=1.3]{B-NAZANIN.TTF}
\setlatintextfont[Scale=1]{DejaVuSans}
\renewcommand{\baselinestretch}{1.5}
\pagestyle{fancy}
\fancyhf{}
\rhead{یادگیری بدون نظارت}
\lhead{\thepage}
%\rfoot{مبنا های عددی}
%\lfoot{9700000}
\renewcommand{\headrulewidth}{1pt}
\renewcommand{\footrulewidth}{1pt}



\begin{document}
	\begin{centering}
		\LARGE
		تمرین اول درس مبانی هوش محاسباتی \\
		\vspace{1em}
		
	\small
		۱۸ اسفند ۱۴۰۲ \\
	\end{centering}
	\section*{مقدمه}
	در این تمرین قصد داریم تا با استفاده از الگوریتم های \lr{clustering} که در کلاس آموختید به خوشه بندی تصاویر افراد مختلف بپردازیم.
	دیتاستی که در این پروژه استفاده شده است، بخشی از دیتاست فلان می‌باشد که شامل فلان عکس می‌باشد. دیتاستی که دراختیار شما قرار گرفته است به این صورت می‌باشد که به ازای هر تصویر،بردارویژگی آن استخراج شده است و برچسب آن نیز در اختیار شما قرار داده شده است.
	
	\section{فاز اول}
	در فاز اول به بررسی دیتا با راهکار های بصری بپردازید.
	
	\section{فاز دوم}
	در فاز دوم با استفاده از الگوریتم های خوشه بندی که در درس آموختید، دیتا را خوشه بندی کنید. انتخاب الگوریتم به عهده شما می‌باشد ولی باید نشان دهید که الگوریتم انتخابی شما بهترین الگوریتم بوده چه از لحاظ عملکرد و چه از لحاظ هایپر‌پارامتر ها.
	\section{فاز سوم}
	در فاز سوم و فاز آخر باید به ارزیابی مدل خود بپردازید
	
	
\end{document}
 
